\documentclass[book_of_proof.tex]{subfiles}
\begin{document}

Em seus estudos verá teoria dos conjuntos incontáveis vezes em diversas campos. Em especial a Análise Real (que é para o que estamos nos preparamos), todo livro que se preze tem um capítulo específico apenas para conjuntos. Isso nos revela o quão relevante e importante é o estudo dos conjuntos.
\par 

Conforme for avançando na matemática verá muitas provas envolvendo conjuntos, álgebra linear, abstrata, análise. Por isso é importante que aprender a ler e compreender provas envolvendo-os.
\par 

Caso seja necessário revise os primeiros capítulos do Projeto Matemática.

\section{Como provar que $ a \in A $}
Sabemos que um conjunto é determinado por uma regra $P(x)$, os elementos do conjunto universo que obedeçam essa $P(x)$ são elementos desse conjunto. Então, dados um conjunto qualquer:
$$A = \{x: P(x)\}$$

Para provar que um $x \in A$ temos que mostrar que $x$ satisfaz a condição $P(x)$.

\textbf{Exemplo:} Prove que $x \in A$ para $x = 5$ onde $A = \lbrace x \in \mathbb{N}: x > 4\rbrace$.
\textit{Prova:} Dados dois números $a$ e $b$, $a \geq b$ se, e somente se, $a - b > 0$. Supondo que $5 \leq 4$ então $5 - 4 \leq 0$, o que claramente é um absurdo. $\blacksquare$

\section{Como provar que $A \subset B$}
Acabamos de ver como provar que um elemento faz parte de um conjunto o que é bem simples. Porém, como provar que um conjunto é subconjunto de outro? Ainda sim, muito simples.
\par

Vamos relembrar que $A$ é um subconjunto de $B$ se todo elemento de $A$ é também elemento de $B$. Em notação matemática:
$$A \subset B \Leftrightarrow \forall a \in A, a \in B$$
Sabendo que o conjunto $A$ é determinado por uma sentença qualquer $P(x)$ e o conjunto $B$ por uma sentença qualquer $Q(x)$, $A \in B \Leftrightarrow [P(x) \Rightarrow Q(x)]$, é importante lembrar que a recíproca não precisa ser verdadeira, já que se dadas todas as condições $B$ é pelo menos tão grande quanto $A$, ou seja, existem elementos em $B$ que não satisfazem $P(x)$.
\\

Então, para provarmos precisamos mostrar que se $x$ satisfaz $P(x)$ também o faz com $Q(x)$, isso para todo $\in A$.

\textbf{Exemplo: } Seja $A = \lbrace x \in \mathbb{N} : 2|x \rbrace$, e $B = \lbrace x \in \mathbb{N} : x^n = 2a \forall a,n \in \mathbb{Z}$. Prove que $A \subset B$:
\\
\textit{Prova:} Se 2 é divisir de $x$ como a primeira sentença determina então $x = 2b, \forall b \in \mathbb{Z}$, ou seja, é um número par. Supondo que existe $n$ tal que $x^n = 2a + 1$ teremos que:
$$(2b)^n = (2b)_1(2b)_2 \dots (2b_n) = 2a + 1$$
Porém se assumirmos que $c = 2^{n-1}b^n$, então podemos reescrever $(2b)^n = 2c$, isso implicaria que:
$$2c = 2a + 1$$

A dúvida que fica: Existe $a$ inteiro que torne isso possível? Bom, não, observe que:
$$ a  = \frac{2c -  1}{2} = c - \frac{1}{2}$$

Porém $c$ é um número inteiro qualquer, então $a \not\in \mathbb{Z}$, o que contradiz a definição do conjunto $B$. Logo, $A \in B$. $\blacksquare$

\section{Provando que $A = B$}

A igualdade entre conjuntos depende que $A \subset B$ e $B \subset A$. Assim, para provar que um conjunto é igual à outro utilizamos o mesmo princípio da seção anterior, porém realizamos para ambas as partes da igualdade.\\
A estrutura seria algo mais ou menos assim:\\

\textbf{Teorema} $A = B$\\
\textit{Prova}: $\forall x \in A$ ... $x \in B$; e
\\
$\forall x \in B$ ... $x \in A$ 
\\
Portanto, $A = B$ $\blacksquare$
\\
Um exemplo para que fique mais claro:

\begin{theorem}
Sejam $A$, $B$ e $C$ conjuntos quaisquer, então:
$$(A - B) \times C = (A \times C) - (B \times C)$$
\end{theorem}
\textit{Prova}:
\begin{align*}
	\begin{aligned}
	(A - B) \times C & = \{(x,y) | x \in (A-B) \land y \in C\} \ \ \ \ \ \ \ \  & \text{(Def. de $\times$)} \\
					 & = \{(x,y) | x \in A \land x \not\in B \land y \in C \} & \text{(Def. de -})  \\
					 & = \{(x,y) | x \in A \land x \not\in B \land y \in C \land y \in C\} & P = P\land P \\
					 & = \{(x,y) | (x \in A \land  y \in C) \land (x \not\in B \land y \in C)\} & \text{Rearranjo} \\
					 & = (A \times C) - (B \times C)			& \text{(Def. de $\times$ e -)} \blacksquare
	\end{aligned}
\end{align*} 














\end{document}